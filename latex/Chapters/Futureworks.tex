\chapter{Future work and internship review}
\lhead{\emph{Future works and internship review}}

In this section we discuss what directions can be taken in the future to further with the physical modeling library, and a review of the internship is presented.

\section{Future work}

What was done during the internship is the beginning of a project more consistent. The ambition is to finally propose a full physical modeling modular library for programmer musician.

First, a lot of modules have to be added to the library. Other kinds of excitation need to be specified for strings, as only a plucking excitation is modelized : struck strings (piano), bowed strings (violin). We also need to code the resonant part of musical instruments, the body, which takes different forms and sizes according to the instrument.
Besides string instruments, with the waveguide model it is possible to model wind instruments (flute, saxophone, oboe, tuba ...). The excitation is a bit different too as you have different manners to blow into the instrument. The end also depends on the concerned instrument so several kinds have to be specified.
All these instrument parts have to consist in a module so there are still a lot to implement. The more modular the library is, the more possibilities it offers for the musician.

Concerning the scripts, the one transforming a geometric object into a modal model has to be reconsidered from the beginning as for me. The idea is really interesting but not easy to do in practise. The use of Elmer may be to reconsidered, as I did not manage to find all information we need for the script to run well, and the help provided on the web was not really useful. It is the counterpart of an open source software. 

\section{Internship review}

First of all, I really appreciated this internship. I looked for a mission which would be part of my professional plan. I am really interested by all music technologies and my co-workers Pierre and Emilio enabled me to choose the subject I prefer, so it was fully benefit for me.

First of all, this internship was really benefit for me as it is a part of my professional plan. I looked for an engineering or research internship with a musical goal. The subject of this internship took into account music, computer science, signal processing and a bit of physics, which is exactly what I wanted.

I discovered the deep part of one kind of sound synthesis, the physical modeling synthesis. As I am interested in the field of sound synthesis I was really glad to work on that topic and I have gathered some knowledge of that. I also discovered the Faust language, which is a really useful language for musical signal processing. I am sure to reuse that in my future career in music computer science. Moreover, I practised a lot and earned some skills in the computer science area : the use of Linux, the python language, some signal processing consideration ...
I am also glad to have discovered the research field. The music computer area is not a huge field and I might work in a research place in my future career. This internship was a nice experience in a computer research center, and I have learned of how the work is done in such a place. I have learned that in a research job you are quite free to organize your work, so it is not that easy to be autonomous and to get things done. You have to keep being motivated by yourself as there is no authority to get you work. My conclusion is that the research area is fantastic if you choose a subject you are fond of, otherwise it will be a pain to make progress in your work.

There also were some more difficult points in this internship, and I learned a lot from them. As I was quite autonomous I had to fight issues by myself, by searching on the Internet or in books. I think it is an good habit to take as a future engineer, we have to find solutions to problems, and it is important to learn how and where to look for. The script \texttt{mesh2dsp.py} was quite a failure because it does not work finally. But I learned that sometimes software we can find on the Internet are not that reliable so we have to do with or find another one.

There is one thing that upset me at the end of my internship. It is that the work in far from being done, as there is still a lot of instrument modules to be implemented. I wish I did more during the internship. Another annoying thing is the impression that all the work done is quite useless, because for now the library is not finished so unusable, and there is no one to continue it full-time. But I put things into perspective and what I understand from that is that when you are in the research area, of course only a couple of people would know or use your work, as it is state-of-the-art technology.

I finally ended my internship with new skills and new knowledge. I have gained experience and vision about music computer field, so that internship was really benefit for me.
