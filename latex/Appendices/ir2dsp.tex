\chapter{\texttt{IR2dsp.py}}
\lhead{\emph{IR2dsp.py}}
\lstset{language=Python}

\begin{lstlisting}
from __future__ import division
import math
import numpy as np
import matplotlib.pyplot as plt
from sys import argv
import subprocess
from scipy.io.wavfile import read
import peakutils
import peakutils.plot as pkplt
import operator
import argparse

#Help for use of the script
parser=argparse.ArgumentParser(
    description="The IR2dsp.py script is a python script that generates a .dsp file from an impulse response file (a .wav file). The impulse response is analyzed in order to rebuild the sound of the vibration of the object based on this impulse response.",
    epilog="See http://faust.grame.fr/images/faust-tutorial2.pdf for more information about Faust")
parser.add_argument('soundFile', type=str, help="Path of the sound file")
parser.add_argument('modelName', help='The name of the model created in the .dsp file')
parser.add_argument('peakThreshold', help='Minimum value of peaks in dB (between - infinity and 0)')
parser.add_argument('peakDistance', help='Minimum distance between two peaks in Hertz')
ars=parser.parse_args()

#Arguments
script, soundFile, modelName, peakThreshold, peakDistance = argv

#Reading file
(fs, x) = read(soundFile)

#Normalizing sound
x = x/max(x)

#FFT
X = np.abs(np.fft.fft(x))
X = X/(max(X))

#computing corresponding frequencies
time_step = 1 / fs
freqs = np.fft.fftfreq(x.size, time_step) 
idx = np.argsort(freqs)

#plot for debug
#plt.plot(freqs[idx], 20*np.log(X[idx]))
#plt.show()

#detecting peaks
threshold = math.pow(10,float(peakThreshold)/20) #from dB to X unit
distance = int(peakDistance)/(fs/x.size)
indexes = peakutils.indexes(X[idx], thres=threshold, min_dist=distance)

#Storing frequencies and modes for each bp filters
peaksFreq = []
peaksGains = []
nbOfPeaks = 0
for i in indexes:
    if freqs[idx][i] > 0:
        peaksFreq.append(freqs[idx][i])
        peaksGains.append(X[idx][i])
        nbOfPeaks += 1
peaksGains = peaksGains/(max(peaksGains))

#Computing t60 values
peakst60 = []
for i in range(0,nbOfPeaks):
    offset = pow(10,-3/20) #conversion of -3dB in X unit
    peakIndex = indexes[len(indexes) - nbOfPeaks + i]

    n = peakIndex
    while X[idx][n] > (X[idx][peakIndex]*offset):
        n = n-1
    a = n

    n = peakIndex
    while X[idx][n] > (X[idx][peakIndex]*offset):
        n = n+1
    b = n
    
    bandwidth = (b-a)/(fs/x.size) #bandwidth in Hz
    print bandwidth
    peakst60.append(6.91 / fs/(1-math.exp(-math.pi*bandwidth/fs)))
    

print "peaks frequencies :"
print peaksFreq
print "corresponding gains :"
print peaksGains
print "corresponding t60 :"
print peakst60

# Writing the dsp file #
########################
file = open(modelName + ".dsp", "w")

file.write("import(\"architecture/pm.lib\");\n")
file.write("import(\"music.lib\");\n\n")
file.write("pi = 4*atan(1.0);\n")
file.write("nModes = ")
file.write(str(len(peaksGains)))
file.write(";\n")
file.write("modeFrequencies = ("); #writing the frequencies list
k = 0
for i in peaksFreq :
    file.write(str(i))
    if(k+1 < len(peaksGains)):
        file.write(", ")
    k += 1
file.write(");\n");
file.write("massEigenValues = ("); #writing the masses list
k = 0
for i in peaksGains :
    file.write(str(i))
    if(k+1 < len(peaksGains)):
        file.write(", ")
    k += 1
file.write(");\n");
file.write("t60 = ("); #writing the t60 list
k = 0
for i in peakst60 :
    file.write(str(i))
    if(k+1 < len(peaksGains)):
        file.write(", ")
    k += 1
file.write(");\n");
file.write("modeFreqs = par(i,nModes,take(i+1,modeFrequencies));\n")
file.write("modeGains = par(i,nModes,take(i+1,massEigenValues));\n")
file.write("modeT60 = par(i,nModes,take(i+1,t60));\n")
file.write(modelName)
file.write(" = modalModel(nModes,modeFrequencies,modeGains,modeT60);");
file.write('\ngate = button("gate");')
file.write('\nprocess = impulseExcitation(gate) : ' + modelName + ' <: _,_;')

file.close();

\end{lstlisting}
