\chapter{Company presentation}
\lhead{\emph{Company Introduction}}

This internship took place at the Centre de recherche en informatique (CRI) which is the computer science laboratory of the MINES ParisTech school. The CRI is located in Fontainebleau, a lovely French former royal city, around 50 km away from Paris.

\section{The MINES ParisTech school}
The MINES ParisTech school is an engineering school, one of the most prestigious in France, created in the year 1783. It provides a high quality education to train new high level engineers. It also hosts an important research activity within its laboratories. The MINES research activity is shared into five research departments : 
\begin{itemize}
    \item Earth science and environment
    \item Energetics and process
    \item Mechanics and materials
    \item Economics, management and society
    \item Mathematics and systems
\end{itemize}

The internship took place in that last department. This department is itself divided into several parts \cite{minesmaths} :
\begin{itemize}
    \item The mathematical morphology center (CMM) whose topic is the science of morphology which is helpful to analyze images.
    \item The robotics center (CAOR) which studies 3D scenes real-time analysis algorithms.
    \item The automatics and systems center (CAS) whose activities are about physical systems control.
    \item The applied mathematics center which shows its skills in modelling and decision theory about climate change.
    \item The computer science research center (CRI) where I did my internship. Its structure will be more detailed in the following section.
    \item The bio-informatics center develops some machine learning methods to analyze biological and chemical data.
\end{itemize}

\section{The Centre de Recherche en Informatique (CRI)}

The CRI is a computer science laboratory located in Fontainebleau which is around 50 km away in the south of Paris. The CRI dedicates its activity to languages for information technology, and develop some techniques for semantics analysis and automated transformations in order to answer to industrial needs (performance, energy consumption, security, development costs) as well as societal and administrative needs (consistent information share, data normalization, information access, heritage safeguard) \cite{criactivity}.

ulIts activity rests upon several current projects in the computer science area, among which we can quote MINDs (a musical therapy for Alzheimer suffering patients), ACOPAL (analysis and compilation of parallel programming languages), or LEGIVOC (it provides a terminology database to help for understanding EU laws).
This internship took part in the FEEVER project.

\section{The FEEVER project}
FEEVER (\textbf{F}aust \textbf{E}nvironment \textbf{Ever}yware) is a project funded by the ANR since October 2013. The motivation of the FEEVER project is the belief that there are a real improvements to make in the audio digital world \cite{feever}. Indeed, today's technologies are not that convenient for the audio engineer. Audio technologies depend on the device on which it is used, and it is not standardize or compatible.

Feever defines itself to be :
\begin{itemize}
    \item portable, to allow program-once, deploy-everywhere economic advantages
    \item easily programmable, to narrow the gap between specifications and implementation,
    \item able to deal with multiple platforms, for seamless integration within the users listening environments,
    \item efficient both in terms of computing time, since audio processing is a highly compute-intensive activity, and energy, if only to permit mobile applications
    \item secure, since audio processing activity performed on the client side must not jeopardize the user system
\end{itemize}
\cite{feever}

